\documentclass[report.tex]{subfiles}

\begin{document}
	
\section{Klasifikačné modely}	
V tejto časti sa budeme venovať binárno-klasifikačným modelom a aplikovať ich na úlohu predpovedania, či sa konkrétna politická strana dostane do parlamentu, alebo nie.

\subsection{Výber algoritmov}
Rozhodli sme sa porovnať tri základné modely, ktoré poskytujú rôzne prístupy ku riešeniu tejto klasifikačnej úlohy:

\begin{itemize}
    \item logistická regresia
    \item rozhodovací strom
    \item support vector machine
\end{itemize}

\subsection{Predspracovanie dát}
Pred trénovaním týchto modelov bolo potrebné získané dáta spracovať do vhodného formátu. Dáta o samotných prieskumoch, ktoré sú už vopred rozdelené na trénovaciu a testovaciu vzorku, majú takúto štruktúru.

\begin{table}[h!]
    \centering
    \caption{Ukážka dát o prieskumoch}
    \begin{tabular}{llcccl}
        \toprule
        \textbf{political\_party} & \textbf{$\ldots$} & \textbf{elected\_to\_parliament} & \textbf{1} & \textbf{2} & \textbf{$\ldots$} \\
        \midrule
        olano    & $\ldots$ & 0        & 8.2      & 6.4 & $\ldots$ \\
        smer\_sd & $\ldots$ & 1        & 34.6     & 38.4 & $\ldots$ \\
        $\vdots$ & $\vdots$ & $\vdots$ & $\vdots$ & $\vdots$ & \\  
        \bottomrule
    \end{tabular}
\end{table}

Teda máme informácie o názve strany, či bola v danom roku naozaj zvolená alebo nie (hodnota 1 označuje, že bola) a údaj v percentách z prieskumu z $k$ mesiacov pred voľbami, kde $k \in \{ 1, \ldots , 12 \}$. Takisto máme pre každé pozorovanie dátum volieb, informáciu o tom či strana bola predtým v koalícii alebo v opozícii a skutočný percentuálny výsledok volieb v danom roku.

Okrem týchto \uv{hlavných} dát, máme dáta aj dáta \uv{vedľajšie}, ktoré zachytávajú všeobecné informácie o sociálno-ekonomickej situácii na Slovensku v danom roku. 


\begin{table}[h!]
    \centering
    \caption{Ekonomické ukazovatele}
    \begin{tabular}{llcccl}
        \toprule
        \textbf{indicator} & \textbf{2010} & \textbf{2011} & $\ldots$ \\
        \midrule
        Unemployment rate (\%) & 12.46    & 13.59  \\
        GDP (per capita)       & 12668    & 13254 \\
        $\vdots$               & $\vdots$ & $\vdots$ \\ 
        \bottomrule
    \end{tabular}
\end{table}

Tieto dve tabuľky sme na základe rokov spojili a následne zaviedli aj nové interakčné premenné. Napríklad sme vynásobili binárnu premennú \textit{in\_coallition\_before} s premennou \textit{Unemployment rate} a ešte niektorými premennými zo všeobecných dát.

Nakoniec sme všetky premenné štandardizovali na tzv. $z$-skóre, teda každá premenná má priemer 0 a štandardnú odchýlku 1. Z takto naškálovaných premenných sme vybrali podmnožinu, na ktorej budeme trénovať naše klasifikačné algoritmy. Menovite to boli tieto premenné.

\begin{python}
    final_variables: list[str] = [
        "1", "2", "3", "4", "5", "6", "7", "8", "9", "10", "11", "12",
        "in_coalition_before_Unemployment rate (%)",
        "in_coalition_before_Pension expenditures (per capita)",
        "in_coalition_before_Expenditures on research and development",
        "Risk of poverty rate", "Total household income", 
        "Total inflation rate (%)", "Gasoline price 95 octane",
    ]
\end{python}

\subsection{Trieda Ensemble}
Okrem hore-uvedených algoritmov sme naimplementovali aj triedu \pyth|Ensemble|, ktorá spája ľubovolné klasifikačné algoritmy, ktoré sa podieľajú štýlom \uv{hlasovania} na finálnej predikcii. 
Parametre potrebné na inicializáciu sú tieto.

\begin{python}
    def __init__(self, models: list[Type[BaseEstimator]],
             metric: Callable, threshold: float = 0.5,
             weights: Optional[list[float]] = None) -> None:
\end{python}
Argument \pyth|models| je zoznam modelov, ktoré budú \uv{hlasovať} o finálnej predikcii. Parameter \pyth|metric| je metrika na základe ktorej sa  proporčne pridelí každému modelu \uv{sila hlasovania}. Hyperparameter \pyth|threshold| určuje, kedy je pozorovanie klasifikované ako \uv{dostane sa do parlamentu} a kedy nie. Voliteľným parametrom \pyth|weights| môžeme dopredu prideliť silu každému modelu.

Po inicializácii môžeme náš \uv{spojený} model natrénovať. Teda trénuje sa osobitne inštancia každého zo vstupných modelov. Nakoniec sa vypočítajú sily predikcií podľa danej metriky, ktoré sa napokon znormalizujú tak, aby dali v súčte $1$. 

Po úspešnom trénovaní, môžeme predikovať takto.
\begin{python}
    def predict(self, X: pd.DataFrame):
    predictions: list[np.ndarray] = []

    for classifier in self.fitted_classifiers:
        predictions.append(np.array(classifier.predict(X=X)))
    
    weighted_sum: np.ndarray = np.dot(
    	np.array(predictions).T, np.array(self.weights)
    	)

    return (weighted_sum >= self.threshold).astype(int)
\end{python}
Vstupom do metódy predict je matica dát a výsledkom je vektor klasifikácií. Po predikcií všetkých modelov sa vypočíta súčin matice predikcií a vektora váh. Nakoniec sa pozložkovo porovná výsledok tohto násobenia so vstupným hyperparametrom \pyth|threshold|.

\subsection{Porovnanie algoritmov}

Pre každý z uvedených prístupov nájdeme doprednou selekciou podmnožinu premenných, ktorá nakoniec použijeme pri finálnom trénovaní. Pre našu triedu \pyth|Ensemble| takisto kros-validáciou nájdeme optimálnu hodnotu hyperparametra \pyth|threshold| pre túto podmnožinu. Nakoniec už môžme len porovnať tieto natrénové modely na testovacích dátach.

\begin{figure}[!htbp]
    \centering
    \includegraphics[width=0.98\textwidth]
    {images/confussion_matrix.png}
    \caption{}
    \label{fig:confussion_matrices}
\end{figure}
Môžeme vidieť, že model \pyth|Ensemle|, v tomto prípade pozostávajúca práve zo všetkých troch modelov, klasifikovala všetky testovacie dáta správne, narozdiel od samotných modelov, kde boli prípady takzvanej falošnej pozitivity. Hoci sa tento výsledok zdá sľubný, testovacie dáta pozostávali len z trinástich pozorovaní. Otázkou teda zostáva, ako by si náš model viedol na väčších dátach.

\end{document}
	