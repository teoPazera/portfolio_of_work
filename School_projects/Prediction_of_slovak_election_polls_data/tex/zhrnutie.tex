\documentclass[report.tex]{subfiles}

\begin{document}
	
\section{Zhrnutie}	

V našej práci sme sa venovali predikovaniu volebných výsledkov za pomoci prieskumov, hodnotového nastavenia strany a socio-ekonomickej situácie štátu. Rozobrali sme si pozorované správanie v dátach, ukázali funkčnosť klasifikovania pokorenia hranice zvoliteľnosti a následne aj predikciu volebného výsledku. Z nie signifikantného zlepšenia testovacej chyby oproti naivnému prediktoru môžeme usúdiť, že voľby sú výrazne komplexnejší proces, ako náš model dokáže zachytiť. Našli sme však prípady, kedy môžeme predpokladať, že strane sa bude dariť lepšie alebo horšie vo voľbách, ako predikujú prieskumy verejnej mienky.

V budúcej práci by sme sa mohli venovať napríklad predikovaniu volieb v iných štátoch, resp. vo viacerých štátoch naraz. Počas celého pracovania na projekte nás limitoval fakt, že dát máme príliš málo na to, aby natrénovaný model zachytil všeobecné správanie. Tiež by sme mohli implementovať do predikcie vývoja preferencií vzájomné ovplyvňovanie sa medzi stranami. Je známy fakt, že voliči sa často medzi podobnými stranami \enquote{prelievajú}, k čomu sú aj dostupné údaje o tzv. druhej voľbe. Takisto pokles popularity koaličných strán by mohol spustiť nástup opozičných strán. 

Na zhodnotenie relevancie našej práce nám ostáva iba zbierať údaje z prieskumov a čakať na najbližšie voľby. Môžeme pozorovať, či sa naše predikcie s blížiacimi sa voľbami ustáľujú alebo kolíšu, a nakoniec ako ďaleko od skutočného výsledku budú. Veríme v pozitívny výsledok.

\end{document}