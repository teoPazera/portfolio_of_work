\documentclass[report.tex]{subfiles}

\begin{document}
	
\section{Predstavenie témy}	
	
V histórii slovenských volieb sme sa stretli už s viacerými prekvapeniami. Či už to bol výsledok strany Smer SD v roku 2012, ktorý im stačil na zloženie jednofarebnej vlády, nečakané prekonanie hranice zvoliteľnosti stranou ĽSNS v roku 2016, ktorá mala mesiac pred voľbami v prieskumoch verejnej mienky okolo 2\%, alebo takisto nečakaná dominantná výhra strany OĽaNO v roku 2020. Tieto javy ukazujú, že prieskumy volebných preferencií sú dobrým odhadom skutočného výsledku volieb, no nezachytávajú v sebe zmenu nálad a názorov v spoločnosti, ktoré nevyhnutne volebný deň prináša.

V našej práci sa budeme venovať práve tomuto. Chceme predikovať výsledky volieb na základe prieskumov verejnej mienky a rôznych informácií o strane či socio-ekomickej situácie na Slovensku. Výsledný model môže slúžiť ako nový odhad výsledkov z volebného dňa, popri odhadovaní samotným posledným prieskumom či exit pollom. 

V nasledujúcich kapitolách sa pozrieme na dáta, s ktorými budeme pracovať. Pokúsime sa opísať nami pozorované správanie v nich a nadizajnujeme klasifikačný a predikčný model. Natrénovaný predikčný model využijeme na odhad výsledku volieb, keby sa konali v decembri 2024, resp. ak by sa konali v máji 2025.
	
\end{document}