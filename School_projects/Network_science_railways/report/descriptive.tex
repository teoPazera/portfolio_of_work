\documentclass[main.tex]{subfiles}

\begin{document}

\section{Deskriptívna analýza}
Ako prvé sa pozrieme na základné štatistické ukazovatele pre obe železničné siete – slovenskú aj českú. Obe tieto siete môžeme vidieť na obrázkoch \ref{obr:1} a \ref{obr:2}.
\begin{figure}
    \centerline{\includegraphics[width=1\textwidth]{images/sk_railroads.png}}
    \caption{Železničná trať Slovenska}
    \label{obr:1}
\end{figure}

\begin{figure}
    \centerline{\includegraphics[width=1\textwidth]{images/cze_railroads.png}}
    \caption{Železničná trať Česka}
    \label{obr:2}
\end{figure}

Okrem iných metrík uvádzame aj \emph{vážený polomer} a \emph{vážený priemer grafu} (tzv.~\emph{diameter}), ktoré bližšie charakterizujú štruktúru siete z hľadiska vzdialeností medzi stanicami.

Tieto metriky definujeme nasledovne:
\begin{align*}
  \text{vážený polomer grafu:}\quad
  \operatorname{rad}(G) &= \min_{v \in V} e(v)
                         = \min_{v \in V} \Bigl(\max_{u \in V} d(v, u)\Bigr), \\[6pt]
  \text{vážený priemer grafu (diameter):}\quad
  \operatorname{diam}(G) &= \max_{v \in V} e(v)
                          = \max_{v \in V} \Bigl(\max_{u \in V} d(v, u)\Bigr),
\end{align*}

\noindent
kde $d(u, v)$ predstavuje váženú vzdialenosť (v našom prípade dĺžku trate) medzi stanicami $u$ a $v$ a $e(v)$ je excentricita vrcholu $v$, teda najväčšia vzdialenosť z tohto vrcholu do ľubovoľného iného vrcholu v grafe.

Základné štatistiky uvádzame v tabuľke \ref{tab:statistics}. Okrem prvých troch metrík uvádzame hodnoty len pre najväčší komponent danáho grafu.

\begin{table}[h!]
\centering
\begin{tabular}{@{}lcc@{}}
\toprule
 & Česko & Slovensko \\
\midrule
Počet komponentov & 35 & 6 \\
Počet všetkých vrcholov (staníc) & 1146 & 426 \\
Počet všetkých hrán (tratí) & 1262 & 449 \\
Počet vrcholov & 1108 & 421 \\
Počet hrán & 1258 & 449 \\
$\mathbb{E}(k)$ & 2{,}27 & 2{,}13 \\
$\rho(G)$ & 0{,}0021 & 0{,}0051 \\
$\operatorname{diam}(G)$ [km] & 613 & 562 \\
$\operatorname{rad}(G)$ [km] & 311 & 282 \\
\bottomrule
\end{tabular}
\caption{Základné štatistiky železničných sietí Slovenska a Česka}
\label{tab:statistics}
\end{table}
Priemerný stupeň vrchola označujeme $\mathbb{E}(k)$ a definujeme ho ako:
\begin{equation*}
    \mathbb{E}(k) = \frac{2m}{n}
\end{equation*}
a $\rho(G)$ označuje hustotu siete.

Ako prvé si môžeme všimnúť rozdielny počet vlakových staníc v Česku a na Slovensku. Tento rozdiel je však do značnej miery ovplyvnený aj rozlohou jednotlivých krajín. Po prepočte na tisíc kilometrov štvorcových získavame hodnoty $14{,}53$ staníc pre Česko a $8{,}69$ pre Slovensko. Znamená to, že Česko má vyššiu hustotu železničných staníc nielen v absolútnych číslach, ale aj v pomere k veľkosti územia.

Na druhej strane však v prípade Česka pozorujeme výrazne väčší počet komponentov súvislosti v železničnej sieti. V tomto bode je potrebné poznamenať, že napriek snahám o čo najpresnejšie doplnenie údajov môže byť takýto vysoký počet komponentov skôr dôsledkom neúplnosti alebo nekonzistencie v dostupných dátach.

V hodnotách priemerného stupňa vrcholu a hustoty siete nepozorujeme výrazné rozdiely medzi oboma krajinami. Naopak, polomer a priemer železničných sietí zodpovedajú veľkostnému rozdielu medzi Českom a Slovenskom, pričom väčšie územie sa prirodzene prejavuje vo väčších hodnotách týchto metrík.

Na obrázku \ref{obr:degree_distrib} je znázornené rozdelenie stupňa vrcholov. V oboch krajinách pozorujeme podobný trend – najčastejšie sa vyskytujú stanice, ktoré sú prepojené s dvomi ďalšími stanicami. Rozdiel však spočíva v tom, že v Česku sa vyskytujú aj stanice so stupňom až päť, čo na Slovensku zaznamenané nebolo.

\begin{figure}
    \centerline{\includegraphics[width=1\textwidth]{images/degree_distrib.png}}
    \caption{Rozdelenie stupňa vrcholov}
    \label{obr:degree_distrib}
\end{figure}

Ako ďalšie si uvedieme porovnanie oboch krajín na základe rôznych centralít.

\textbf{Centralita prepojenosti} -- táto metrika nám hovorí o tom na koľko je daný uzol prepojovacím článkom v celej sieti, teda koľko dôležitých ciest cezeň vedie. Formálne pre každý vrchol $v$ definujeme 
\begin{equation*}
    c_B(v) =\frac{\sum_{s,t \in V} \frac{\sigma(s, t|v)}{\sigma(s, t)}}{\frac{(n-1)(n-2)}{2}},
\end{equation*}
kde $V$ je množina uzlov, $\sigma(s, t)$ je počet najkratších ciest medzi $s$ a $t$ a $\sigma(s, t |v)$ je počet týchto ciest, ktoré vedú cez nejaký uzol $v$ odlišný od $s$ a $t$. Ak $s = t$, potom $\sigma(s, t) = 1$, a ak $v \in \{s, t \}$, potom $\sigma(s, t |v ) = 0$. 

\begin{figure}
    \centerline{\includegraphics[width=1\textwidth]{images/betweenes.png}}
    \caption{Centralita prepojenosti}
    \label{obr:betweenes}
\end{figure}

Situáciu našich dát možno vidieť na obrázku \ref{obr:betweenes}. Najdôležitejšie stanice z hľadiska centrality prepojenosti na Slovensku sa nachádzajú na „strednej“ trati, konkrétne na úseku vedúcom z Nových Zámkov cez Banskú Bystricu až do Prešova. Tento trend je očakávaný, keďže severná a južná trať sú oproti nej obchádzkovejšie a teda menej priame. V Česku pozorujeme podobný vzor: kľúčové uzly sa sústreďujú v strednej časti republiky, napríklad na trasách z Prahy cez Pardubice alebo z Prahy južnejšie smerom do Brna.

\textbf{Katzova centralita} -- metrika, ktorá rožširuje ideu centrality vlastného vektora tak, že berie do úvahy nielen priame susedstvá,  ale aj všetky dlhšie cesty v grafe, pričom príspevok každej cesty klesá s jej dĺžkou. Pre vrchol $i$ ju definujeme ako:
\begin{equation*}
    x_i = \alpha \sum_{j} A_{ij} x_j + \beta,
\end{equation*}
kde $A$ je matica susedností s vlastnými hodnotami $\lambda$.
Parameter $\beta$ určuje počiatočnú hodnotu centrality a platí
\begin{equation*}
    \alpha < \frac{1}{\lambda_{\max}}.    
\end{equation*}

Pre parameter $\beta$ sme zvolili hodnotu $1$ a $\alpha$ sme nastavili na $\frac{0.8}{\lambda_{max}}$. Na obrázku \ref{obr:katz} môžeme vidieť výsledky pre obe krajiny.

\begin{figure}
    \centerline{\includegraphics[width=1\textwidth]{images/katz.png}}
    \caption{Katzova centralita}
    \label{obr:katz}
\end{figure}

Vidíme, že sa pre Slovensko aj Česko ukazujú dôležité stanice v hlavných mestách.
Pre Česko tento výsledok nie je až taký prekvapivý, keďže Praha má okolo seba veľa zastávok. Na druhej strane však podobne dôležitou vychádza aj Bratislava, ktorá má v tesnej blízkosti menší počet zastávok.
Taktiež si môžeme všimnúť, že Bratislava je podľa tejto metriky na rovnakej úrovni ako napríklad Vrútky alebo Šurany, ktoré sú pomerne významnými prestupnými stanicami.
Okrem toho vidíme na Slovensku aj pomerne významnú časť severnej trate vedúcej cez Trenčín.

Ako posledné sa pozrieme na skupiny vrcholov a to v južnej a severnej časti krajiny, keďže trate sa zdajú byť celkom odseparované. Sieť si najprv rozdelíme na severnú a južnú časť, tak ako na obrázku \ref{obr:split}.

\begin{figure}
    \centerline{\includegraphics[width=1\textwidth]{images/south_north_split.png}}
    \caption{Rozdelenie na južnú a severnú časť pre obe krajiny}
    \label{obr:split}
\end{figure}

Pre takto rozdelenú sieť spočítame modularitu vzhľadom na príslušnosť ku danej skupine. Váženú modularitu vzhľadom na nezoraditeľnú vlastnosť definujeme takto: 
 \[
Q_w \;=\;
\frac{1}{2m}
\sum_{i,j}
\Bigl(
  w_{ij}
  - \frac{k_i\,k_j}{2m}
\Bigr)
\;\delta_{g_i,g_j},
\]
kde
\[
k_i = \sum_j w_{ij},
\qquad
m = \frac{1}{2}\sum_{i,j}w_{ij},
\qquad
\delta_{g_i,g_j} =
\begin{cases}
1, & g_i = g_j,\\
0, & \text{inak.}
\end{cases}
\]

Vyšli nám hodnoty $0.96$ pre Češko a $0.90$ pre Slovensko čo znamená, že väčšina váhy, teda kilometrov trate sa nachádza vnútri daných skupín. Prekvapivo pre Česko je to o $6\%$ viac ako pre Slovensko, čo by sme nemuseli očakávať, keďže na obrázku \ref{obr:split} vidíme, že vrcholy sú na nami zadefinovanej hranici oddeľujúcej \uv{juh} a \uv{sever} hustejšie, teda by sme mohli očakávať viacero prepojení (tratí), medzi týmito mestami.
\end{document}

