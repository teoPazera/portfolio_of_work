\documentclass[english]{beamer}

\usepackage[utf8]{inputenc}

\usepackage{amsmath, amsfonts}
\usepackage[slovak]{babel}
\usepackage{caption}
\usepackage{csquotes}
\usepackage[T1]{fontenc}
\usepackage{graphicx}
\usepackage{hyperref}
\usepackage{mathtools}
\usepackage{multimedia}
\usepackage{multirow}
\usepackage{newalg, wasysym, slashbox,lscape}
\usepackage{subcaption}
\usepackage{tikz}
\usetikzlibrary{arrows.meta}
\usepackage{times}
\usepackage{xcolor}
\usepackage{booktabs}
%\usepackage{epsfig}
%\usepackage{pgf,pgfarrows,pgfnodes,pgfautomata,pgfheaps}

\mode<presentation>
{
  \usetheme{Madrid}  

  \setbeamercovered{transparent}
}

\makeatletter
\setbeamertemplate{footline}
{
	\leavevmode%
	\hbox{%
		\begin{beamercolorbox}[wd=.6\paperwidth,ht=2.25ex,dp=1ex,center]{title in head/foot}%
			\usebeamerfont{title in head/foot}\insertshorttitle
		\end{beamercolorbox}%
		\begin{beamercolorbox}[wd=.25\paperwidth,ht=2.25ex,dp=1ex,right]{date in head/foot}%
			\usebeamerfont{date in head/foot}\insertshortdate{}\hspace*{2ex}
		\end{beamercolorbox}%
		\begin{beamercolorbox}[wd=.15\paperwidth,ht=2.25ex,dp=1ex,right]{date in head/foot}%
			\usebeamerfont{date in head/foot}\insertframenumber{} / \inserttotalframenumber\hspace*{2ex}
		\end{beamercolorbox}%
	}%
	\vskip0pt%
}
\makeatother

\captionsetup{labelformat=empty, justification=centering, font=footnotesize}

\definecolor{lightgray}{gray}{0.8}
\newcommand{\fade}[1]{\textcolor{lightgray}{\ensuremath{#1}}}

\newcommand{\mycite}[1]{\tiny [#1] \normalsize}

\beamerdefaultoverlayspecification{}
\setbeamertemplate{navigation symbols}{}
\setbeamertemplate{blocks}[rounded][shadow=false]

\setbeamerfont{title}{series=\bfseries} % Make the title bold
\setbeamerfont{frametitle}{series=\bfseries} % Make the frametitle bold


\title{Porovnanie kvality slovenskej a českej železničnej infraštruktúry}
\author{Tomáš Antal, Teo Pazera, Andrej Špitalský, 3DAV}
\institute[]{
	Fakulta matematiky, fyziky a informatiky\\
	Univerzity Komenského, Bratislava
}
\date{\today}

\begin{document}

\begin{frame}
  \titlepage
\end{frame}

\begin{frame}
  	\frametitle{Porovnanie kvality železničných sietí}
	
	\begin{block}{}
		\begin{itemize}
			\item hypotéza: Česko má lepšie vybudovanú železničnú sieť, ako Slovensko
			\item cieľ: kvantifikovať rozdiely a identifikovať slabé miesta
		\end{itemize}	
	\end{block}
	
	\begin{columns}[T]
		\column{0.5\textwidth}
		\centering
		\includegraphics[width=0.95\columnwidth]{presentation_images/cd.jpg}
		
		\column{0.5\textwidth}
		\centering
		\includegraphics[width=0.95\columnwidth]{presentation_images/zssk.jpg}
	\end{columns}
	
\end{frame}

\begin{frame}
	\frametitle{Dáta}
	
	\begin{block}{}
		\begin{itemize}
			\item zdroje
			\begin{itemize}
				\item Humanitarian OpenStreetMap (železničné stanice a spojenia)
				\item GitHub a Český Statistický Úřad (obyvateľstvá miest a dedín)
			\end{itemize}	
			\item graf pre každú železničnú sieť
			\begin{itemize}
				\item vrcholy: stanice
				\item hrany: ak existuje spojenie medzi dvoma stanicami, váhované vzdušnou vzdialenosťou
			\end{itemize}
		\end{itemize}
	\end{block}
	
	\centering
	\includegraphics[width=0.75\textwidth]{presentation_images/combined_map_final_edited.png}
\end{frame}

\begin{frame}
	\frametitle{Priemerný stupeň a hustota}
	
	\begin{table}
		\centering
		\begin{tabular}{@{}lcc@{}}
			\toprule
			& Česko & Slovensko \\
			\midrule
			počet vrcholov & 1108 & 421 \\
			počet hrán & 1258 & 449 \\
			$\mathbb{E}(k)$ & 2{,}27 & 2{,}13 \\
			$\rho(G)$ & 0{,}0021 & 0{,}0051 \\
			\bottomrule
		\end{tabular}
	\end{table}
	
	\centering
	\includegraphics[width=0.95\textwidth]{presentation_images/degree_distrib_edited.png}
\end{frame}

\begin{frame}
	\frametitle{Centralita prepojenosti}
	
	\centering
	\includegraphics[width=\textwidth]{presentation_images/betweenes_edited.png}
\end{frame}

\begin{frame}
	\frametitle{Katzova centralita}
	
	\centering
	\includegraphics[width=\textwidth]{presentation_images/katz_edited.png}
\end{frame}


\begin{frame}
	\frametitle{Pomer najkratšej cesty so vzdušnou čiarou}
	
	\begin{minipage}[t][0.3\textheight][t]{\textwidth}
		\only<1>{
			\begin{block}{}
				\begin{itemize}
					\item ako veľkú obchádzku musí náhodný cestujúci robiť, oproti vzdušnej čiare?
					\item generovanie cestujúcich (začiatočná a koncová stanica) váhované počtami obyvateľov v prislúchajúcej obci/mestu
				\end{itemize}	
			\end{block}
		}
		\only<2>{			
			\begin{table}[h!]
				\centering
				\begin{tabular}{@{}lcc@{}}
					\toprule
					& Česko & Slovensko \\
					\midrule
					priemerný pomer &  1.3394 & 1.7941 \\
					medián pomeru & 1.2710 & 1.3172 \\
					maximálny pomer & 38.3310 & 375.0649 \\
					\bottomrule
				\end{tabular}
			\end{table}
			
		}
	\end{minipage}
	
	\begin{minipage}[t][0.7\textheight][t]{\textwidth}
		\vspace{0.3cm}
		\begin{columns}[T]
			\column{0.5\textwidth}
			\centering
			\includegraphics[width=0.95\columnwidth]{presentation_images/cz_shortest_path_vs_skyline_edited.png}
			
			\column{0.5\textwidth}
			\centering
			\includegraphics[width=0.95\columnwidth]{presentation_images/sk_shortest_path_vs_skyline_edited.png}
		\end{columns}
	\end{minipage}
\end{frame}

\begin{frame}
	\frametitle{Kirchhoffov index}
	
	\begin{block}{}
		\begin{itemize}
			\item globálna rezistentnosť grafu -- komplikovanosť dopravy
		\end{itemize}	
	\end{block}
	
	\begin{equation*}
		\mathit{Kf}(G) = \sum_{\substack{i, j \in V(G) \\ i < j}} R_{i,j} = n\sum_{i=1}^{n-1} \frac{1}{\lambda_i},
	\end{equation*}
	
	\begin{block}{}
		\begin{itemize}
			\item normalizácia pre porovnávanie
		\end{itemize}	
	\end{block}
	
	\begin{equation*}
		\mathit{Kf}_n(G) = \frac{1}{\binom{n}{2}} \mathit{Kf}(G)
	\end{equation*}
	
	
	\begin{align*}
		\mathit{Kf}_n(G_{\text{cze}}) &= 65.99 & \mathit{Kf}_n(G_{\text{svk}}) &= 92.29
	\end{align*}
\end{frame}

\begin{frame}
	\frametitle{Centralita efektívnej rezistencie}	
	\begin{minipage}[t][0.15\textheight][t]{\textwidth}
		\begin{block}{}
			\begin{itemize}
				\item relatívna zmena $\mathit{Kf}_n(G)$ po odstránení hrany/vrchola
			\end{itemize}	
		\end{block}
	\end{minipage}
	
	\begin{minipage}[t][0.85\textheight][t]{\textwidth}
		\only<1>{
			\centering
			\includegraphics[width=0.95\columnwidth]{../images/svk_edge_resistance.png}
		}
		\only<2>{
			\centering
			\includegraphics[width=0.85\columnwidth]{../images/cze_edge_resistance.png}
		}
	\end{minipage}
\end{frame}

\begin{frame}
	\frametitle{Centralita efektívnej rezistencie -- rozdelenie}
	
	\centering
	\includegraphics[width=0.85\columnwidth]{../images/edge_resistance_distribution.png}
\end{frame}

\begin{frame}
	\frametitle{Výsledky}
	\begin{block}{}
		\begin{itemize}
			\item podobná hustota a rozdelenie stupňov vrcholov
			\item železničné trasy v Česku sú efektívnejšie -- bližšie kopírujú vzdušnú čiaru
			\item rezistencia dopravy po českej železničnej siete je nižšia
			\item slovenská železničná sieť je náchylnejšia na výpadok
		\end{itemize}	
	\end{block}
	\vspace{0.2cm}
	\centering
	\includegraphics[width=0.8\columnwidth]{presentation_images/cd_shorter.jpg}
\end{frame}

\begin{frame}
	\frametitle{Ďakujeme za pozornosť}
	
	\centering
	\includegraphics[width=0.95\columnwidth]{presentation_images/meskanie.jpg}
	
\end{frame}




\end{document}
